\section{Combinatorial optimization}
\subsection{Sparse max-flow (C++)}
\begin{lstlisting}[language=C++]
// Adjacency list implementation of Dinic's blocking flow algorithm.
// This is very fast in practice, and only loses to push-relabel flow.
//
// Running time:
//     O(|V|^2 |E|)
//
// INPUT:
//     - graph, constructed using AddEdge()
//     - source and sink
//
// OUTPUT:
//     - maximum flow value
//     - To obtain actual flow values, look at edges with capacity > 0
//       (zero capacity edges are residual edges).

#include <iostream>
#include <vector>

using namespace std;
typedef long long LL;

struct Edge {
  int from, to, cap, flow, index;
  Edge(int from, int to, int cap, int flow, int index) :
    from(from), to(to), cap(cap), flow(flow), index(index) {}
  LL rcap() { return cap - flow; }
};

struct Dinic {
  int N;
  vector<vector<Edge> > G;
  vector<vector<Edge *> > Lf;
  vector<int> layer;
  vector<int> Q;
  
  Dinic(int N) : N(N), G(N), Q(N) {}
  
  void AddEdge(int from, int to, int cap) {
    if (from == to) return;
    G[from].push_back(Edge(from, to, cap, 0, G[to].size()));
    G[to].push_back(Edge(to, from, 0, 0, G[from].size() - 1));
  }

  LL BlockingFlow(int s, int t) {
    layer.clear(); layer.resize(N, -1);
    layer[s] = 0;
    Lf.clear(); Lf.resize(N);
    
    int head = 0, tail = 0;
    Q[tail++] = s;
    while (head < tail) {
      int x = Q[head++];
      for (int i = 0; i < G[x].size(); i++) {
        Edge &e = G[x][i]; if (e.rcap() <= 0) continue;
        if (layer[e.to] == -1) {
          layer[e.to] = layer[e.from] + 1;
          Q[tail++] = e.to;
        }
        if (layer[e.to] > layer[e.from]) {
          Lf[e.from].push_back(&e);
        }
      }
    }
    if (layer[t] == -1) return 0;

    LL totflow = 0;
    vector<Edge *> P;
    while (!Lf[s].empty()) {
      int curr = P.empty() ? s : P.back()->to;
      if (curr == t) { // Augment
        LL amt = P.front()->rcap();
        for (int i = 0; i < P.size(); ++i) {
          amt = min(amt, P[i]->rcap());
        }
        totflow += amt;
        for (int i = P.size() - 1; i >= 0; --i) {
          P[i]->flow += amt;
          G[P[i]->to][P[i]->index].flow -= amt;
          if (P[i]->rcap() <= 0) {
            Lf[P[i]->from].pop_back();
            P.resize(i);
          }
        }
      } else if (Lf[curr].empty()) { // Retreat
        P.pop_back();
        for (int i = 0; i < N; ++i)
          for (int j = 0; j < Lf[i].size(); ++j)
            if (Lf[i][j]->to == curr)
              Lf[i].erase(Lf[i].begin() + j);
      } else { // Advance
        P.push_back(Lf[curr].back());
      }
    }
    return totflow;
  }

  LL GetMaxFlow(int s, int t) {
    LL totflow = 0;
    while (LL flow = BlockingFlow(s, t))
      totflow += flow;
    return totflow;
  }
};

\end{lstlisting}
\subsection{Min-cost max-flow (C++)}
\begin{lstlisting}[language=C++]
// Implementation of min cost max flow algorithm using adjacency
// matrix (Edmonds and Karp 1972).  This implementation keeps track of
// forward and reverse edges separately (so you can set cap[i][j] !=
// cap[j][i]).  For a regular max flow, set all edge costs to 0.
//
// Running time, O(|V|^2) cost per augmentation
//     max flow:           O(|V|^3) augmentations
//     min cost max flow:  O(|V|^4 * MAX_EDGE_COST) augmentations
//     
// INPUT: 
//     - graph, constructed using AddEdge()
//     - source
//     - sink
//
// OUTPUT:
//     - (maximum flow value, minimum cost value)
//     - To obtain the actual flow, look at positive values only.

#include <cmath>
#include <vector>
#include <iostream>

using namespace std;

typedef vector<int> VI;
typedef vector<VI> VVI;
typedef long long L;
typedef vector<L> VL;
typedef vector<VL> VVL;
typedef pair<int, int> PII;
typedef vector<PII> VPII;

const L INF = numeric_limits<L>::max() / 4;

struct MinCostMaxFlow {
  int N;
  VVL cap, flow, cost;
  VI found;
  VL dist, pi, width;
  VPII dad;

  MinCostMaxFlow(int N) : 
    N(N), cap(N, VL(N)), flow(N, VL(N)), cost(N, VL(N)), 
    found(N), dist(N), pi(N), width(N), dad(N) {}
  
  void AddEdge(int from, int to, L cap, L cost) {
    this->cap[from][to] = cap;
    this->cost[from][to] = cost;
  }
  
  void Relax(int s, int k, L cap, L cost, int dir) {
    L val = dist[s] + pi[s] - pi[k] + cost;
    if (cap && val < dist[k]) {
      dist[k] = val;
      dad[k] = make_pair(s, dir);
      width[k] = min(cap, width[s]);
    }
  }

  L Dijkstra(int s, int t) {
    fill(found.begin(), found.end(), false);
    fill(dist.begin(), dist.end(), INF);
    fill(width.begin(), width.end(), 0);
    dist[s] = 0;
    width[s] = INF;
    
    while (s != -1) {
      int best = -1;
      found[s] = true;
      for (int k = 0; k < N; k++) {
        if (found[k]) continue;
        Relax(s, k, cap[s][k] - flow[s][k], cost[s][k], 1);
        Relax(s, k, flow[k][s], -cost[k][s], -1);
        if (best == -1 || dist[k] < dist[best]) best = k;
      }
      s = best;
    }

    for (int k = 0; k < N; k++)
      pi[k] = min(pi[k] + dist[k], INF);
    return width[t];
  }

  pair<L, L> GetMaxFlow(int s, int t) {
    L totflow = 0, totcost = 0;
    while (L amt = Dijkstra(s, t)) {
      totflow += amt;
      for (int x = t; x != s; x = dad[x].first) {
        if (dad[x].second == 1) {
          flow[dad[x].first][x] += amt;
          totcost += amt * cost[dad[x].first][x];
        } else {
          flow[x][dad[x].first] -= amt;
          totcost -= amt * cost[x][dad[x].first];
        }
      }
    }
    return make_pair(totflow, totcost);
  }
};

\end{lstlisting}
\subsection{Push-relabel max-flow (C++)}
\begin{lstlisting}[language=C++]
// Adjacency list implementation of FIFO push relabel maximum flow
// with the gap relabeling heuristic.  This implementation is
// significantly faster than straight Ford-Fulkerson.  It solves
// random problems with 10000 vertices and 1000000 edges in a few
// seconds, though it is possible to construct test cases that
// achieve the worst-case.
//
// Running time:
//     O(|V|^3)
//
// INPUT: 
//     - graph, constructed using AddEdge()
//     - source
//     - sink
//
// OUTPUT:
//     - maximum flow value
//     - To obtain the actual flow values, look at all edges with
//       capacity > 0 (zero capacity edges are residual edges).

#include <cmath>
#include <vector>
#include <iostream>
#include <queue>

using namespace std;

typedef long long LL;

struct Edge {
  int from, to, cap, flow, index;
  Edge(int from, int to, int cap, int flow, int index) :
    from(from), to(to), cap(cap), flow(flow), index(index) {}
};

struct PushRelabel {
  int N;
  vector<vector<Edge> > G;
  vector<LL> excess;
  vector<int> dist, active, count;
  queue<int> Q;

  PushRelabel(int N) : N(N), G(N), excess(N), dist(N), active(N), count(2*N) {}

  void AddEdge(int from, int to, int cap) {
    G[from].push_back(Edge(from, to, cap, 0, G[to].size()));
    if (from == to) G[from].back().index++;
    G[to].push_back(Edge(to, from, 0, 0, G[from].size() - 1));
  }

  void Enqueue(int v) { 
    if (!active[v] && excess[v] > 0) { active[v] = true; Q.push(v); } 
  }

  void Push(Edge &e) {
    int amt = int(min(excess[e.from], LL(e.cap - e.flow)));
    if (dist[e.from] <= dist[e.to] || amt == 0) return;
    e.flow += amt;
    G[e.to][e.index].flow -= amt;
    excess[e.to] += amt;    
    excess[e.from] -= amt;
    Enqueue(e.to);
  }
  
  void Gap(int k) {
    for (int v = 0; v < N; v++) {
      if (dist[v] < k) continue;
      count[dist[v]]--;
      dist[v] = max(dist[v], N+1);
      count[dist[v]]++;
      Enqueue(v);
    }
  }

  void Relabel(int v) {
    count[dist[v]]--;
    dist[v] = 2*N;
    for (int i = 0; i < G[v].size(); i++) 
      if (G[v][i].cap - G[v][i].flow > 0)
  dist[v] = min(dist[v], dist[G[v][i].to] + 1);
    count[dist[v]]++;
    Enqueue(v);
  }

  void Discharge(int v) {
    for (int i = 0; excess[v] > 0 && i < G[v].size(); i++) Push(G[v][i]);
    if (excess[v] > 0) {
      if (count[dist[v]] == 1) 
  Gap(dist[v]); 
      else
  Relabel(v);
    }
  }

  LL GetMaxFlow(int s, int t) {
    count[0] = N-1;
    count[N] = 1;
    dist[s] = N;
    active[s] = active[t] = true;
    for (int i = 0; i < G[s].size(); i++) {
      excess[s] += G[s][i].cap;
      Push(G[s][i]);
    }
    
    while (!Q.empty()) {
      int v = Q.front();
      Q.pop();
      active[v] = false;
      Discharge(v);
    }
    
    LL totflow = 0;
    for (int i = 0; i < G[s].size(); i++) totflow += G[s][i].flow;
    return totflow;
  }
};

\end{lstlisting}
\subsection{Min-cost matching (C++)}
\begin{lstlisting}[language=C++]
//////////////////////////////////////////////////////////////////////
// Min cost bipartite matching via shortest augmenting paths
//
// This is an O(n^3) implementation of a shortest augmenting path
// algorithm for finding min cost perfect matchings in dense
// graphs.  In practice, it solves 1000x1000 problems in around 1
// second.
//
//   cost[i][j] = cost for pairing left node i with right node j
//   Lmate[i] = index of right node that left node i pairs with
//   Rmate[j] = index of left node that right node j pairs with
//
// The values in cost[i][j] may be positive or negative.  To perform
// maximization, simply negate the cost[][] matrix.
//////////////////////////////////////////////////////////////////////

#include <algorithm>
#include <cstdio>
#include <cmath>
#include <vector>

using namespace std;

typedef vector<double> VD;
typedef vector<VD> VVD;
typedef vector<int> VI;

double MinCostMatching(const VVD &cost, VI &Lmate, VI &Rmate) {
  int n = int(cost.size());

  // construct dual feasible solution
  VD u(n);
  VD v(n);
  for (int i = 0; i < n; i++) {
    u[i] = cost[i][0];
    for (int j = 1; j < n; j++) u[i] = min(u[i], cost[i][j]);
  }
  for (int j = 0; j < n; j++) {
    v[j] = cost[0][j] - u[0];
    for (int i = 1; i < n; i++) v[j] = min(v[j], cost[i][j] - u[i]);
  }
  
  // construct primal solution satisfying complementary slackness
  Lmate = VI(n, -1);
  Rmate = VI(n, -1);
  int mated = 0;
  for (int i = 0; i < n; i++) {
    for (int j = 0; j < n; j++) {
      if (Rmate[j] != -1) continue;
      if (fabs(cost[i][j] - u[i] - v[j]) < 1e-10) {
  Lmate[i] = j;
  Rmate[j] = i;
  mated++;
  break;
      }
    }
  }
  
  VD dist(n);
  VI dad(n);
  VI seen(n);
  
  // repeat until primal solution is feasible
  while (mated < n) {
    
    // find an unmatched left node
    int s = 0;
    while (Lmate[s] != -1) s++;
    
    // initialize Dijkstra
    fill(dad.begin(), dad.end(), -1);
    fill(seen.begin(), seen.end(), 0);
    for (int k = 0; k < n; k++) 
      dist[k] = cost[s][k] - u[s] - v[k];
    
    int j = 0;
    while (true) {
      
      // find closest
      j = -1;
      for (int k = 0; k < n; k++) {
  if (seen[k]) continue;
  if (j == -1 || dist[k] < dist[j]) j = k;
      }
      seen[j] = 1;
      
      // termination condition
      if (Rmate[j] == -1) break;
      
      // relax neighbors
      const int i = Rmate[j];
      for (int k = 0; k < n; k++) {
  if (seen[k]) continue;
  const double new_dist = dist[j] + cost[i][k] - u[i] - v[k];
  if (dist[k] > new_dist) {
    dist[k] = new_dist;
    dad[k] = j;
  }
      }
    }
    
    // update dual variables
    for (int k = 0; k < n; k++) {
      if (k == j || !seen[k]) continue;
      const int i = Rmate[k];
      v[k] += dist[k] - dist[j];
      u[i] -= dist[k] - dist[j];
    }
    u[s] += dist[j];
    
    // augment along path
    while (dad[j] >= 0) {
      const int d = dad[j];
      Rmate[j] = Rmate[d];
      Lmate[Rmate[j]] = j;
      j = d;
    }
    Rmate[j] = s;
    Lmate[s] = j;
    
    mated++;
  }
  
  double value = 0;
  for (int i = 0; i < n; i++)
    value += cost[i][Lmate[i]];
  
  return value;
}

\end{lstlisting}
\subsection{Max bipartite matching (C++)}
\begin{lstlisting}[language=C++]
// This code performs maximum bipartite matching.
//
// Running time: O(|E| |V|) -- often much faster in practice
//
//   INPUT: w[i][j] = edge between row node i and column node j
//   OUTPUT: mr[i] = assignment for row node i, -1 if unassigned
//           mc[j] = assignment for column node j, -1 if unassigned
//           function returns number of matches made

#include <vector>

using namespace std;

typedef vector<int> VI;
typedef vector<VI> VVI;

bool FindMatch(int i, const VVI &w, VI &mr, VI &mc, VI &seen) {
  for (int j = 0; j < w[i].size(); j++) {
    if (w[i][j] && !seen[j]) {
      seen[j] = true;
      if (mc[j] < 0 || FindMatch(mc[j], w, mr, mc, seen)) {
        mr[i] = j;
        mc[j] = i;
        return true;
      }
    }
  }
  return false;
}

int BipartiteMatching(const VVI &w, VI &mr, VI &mc) {
  mr = VI(w.size(), -1);
  mc = VI(w[0].size(), -1);
  
  int ct = 0;
  for (int i = 0; i < w.size(); i++) {
    VI seen(w[0].size());
    if (FindMatch(i, w, mr, mc, seen)) ct++;
  }
  return ct;
}

\end{lstlisting}
\subsection{Global min cut (C++)}
\begin{lstlisting}[language=C++]
// Adjacency matrix implementation of Stoer-Wagner min cut algorithm.
//
// Running time:
//     O(|V|^3)
//
// INPUT: 
//     - graph, constructed using AddEdge()
//
// OUTPUT:
//     - (min cut value, nodes in half of min cut)

#include <cmath>
#include <vector>
#include <iostream>

using namespace std;

typedef vector<int> VI;
typedef vector<VI> VVI;

const int INF = 1000000000;

pair<int, VI> GetMinCut(VVI &weights) {
  int N = weights.size();
  VI used(N), cut, best_cut;
  int best_weight = -1;
  
  for (int phase = N-1; phase >= 0; phase--) {
    VI w = weights[0];
    VI added = used;
    int prev, last = 0;
    for (int i = 0; i < phase; i++) {
      prev = last;
      last = -1;
      for (int j = 1; j < N; j++)
  if (!added[j] && (last == -1 || w[j] > w[last])) last = j;
      if (i == phase-1) {
  for (int j = 0; j < N; j++) weights[prev][j] += weights[last][j];
  for (int j = 0; j < N; j++) weights[j][prev] = weights[prev][j];
  used[last] = true;
  cut.push_back(last);
  if (best_weight == -1 || w[last] < best_weight) {
    best_cut = cut;
    best_weight = w[last];
  }
      } else {
  for (int j = 0; j < N; j++)
    w[j] += weights[last][j];
  added[last] = true;
      }
    }
  }
  return make_pair(best_weight, best_cut);
}

\end{lstlisting}
\subsection{Graph cut inference (C++)}
\begin{lstlisting}[language=C++]
// Special-purpose {0,1} combinatorial optimization solver for
// problems of the following by a reduction to graph cuts:
//
//        minimize         sum_i  psi_i(x[i]) 
//  x[1]...x[n] in {0,1}      + sum_{i < j}  phi_{ij}(x[i], x[j])
//
// where
//      psi_i : {0, 1} --> R
//   phi_{ij} : {0, 1} x {0, 1} --> R
//
// such that
//   phi_{ij}(0,0) + phi_{ij}(1,1) <= phi_{ij}(0,1) + phi_{ij}(1,0)  (*)
//
// This can also be used to solve maximization problems where the
// direction of the inequality in (*) is reversed.
//
// INPUT: phi -- a matrix such that phi[i][j][u][v] = phi_{ij}(u, v)
//        psi -- a matrix such that psi[i][u] = psi_i(u)
//        x -- a vector where the optimal solution will be stored
//
// OUTPUT: value of the optimal solution
//
// To use this code, create a GraphCutInference object, and call the
// DoInference() method.  To perform maximization instead of minimization,
// ensure that #define MAXIMIZATION is enabled.

#include <vector>
#include <iostream>

using namespace std;

typedef vector<int> VI;
typedef vector<VI> VVI;
typedef vector<VVI> VVVI;
typedef vector<VVVI> VVVVI;

const int INF = 1000000000;

// comment out following line for minimization
#define MAXIMIZATION

struct GraphCutInference {
  int N;
  VVI cap, flow;
  VI reached;
  
  int Augment(int s, int t, int a) {
    reached[s] = 1;
    if (s == t) return a; 
    for (int k = 0; k < N; k++) {
      if (reached[k]) continue;
      if (int aa = min(a, cap[s][k] - flow[s][k])) {
  if (int b = Augment(k, t, aa)) {
    flow[s][k] += b;
    flow[k][s] -= b;
    return b;
  }
      }
    }
    return 0;
  }
  
  int GetMaxFlow(int s, int t) {
    N = cap.size();
    flow = VVI(N, VI(N));
    reached = VI(N);
    
    int totflow = 0;
    while (int amt = Augment(s, t, INF)) {
      totflow += amt;
      fill(reached.begin(), reached.end(), 0);
    }
    return totflow;
  }
  
  int DoInference(const VVVVI &phi, const VVI &psi, VI &x) {
    int M = phi.size();
    cap = VVI(M+2, VI(M+2));
    VI b(M);
    int c = 0;

    for (int i = 0; i < M; i++) {
      b[i] += psi[i][1] - psi[i][0];
      c += psi[i][0];
      for (int j = 0; j < i; j++)
  b[i] += phi[i][j][1][1] - phi[i][j][0][1];
      for (int j = i+1; j < M; j++) {
  cap[i][j] = phi[i][j][0][1] + phi[i][j][1][0] - phi[i][j][0][0] - phi[i][j][1][1];
  b[i] += phi[i][j][1][0] - phi[i][j][0][0];
  c += phi[i][j][0][0];
      }
    }
    
#ifdef MAXIMIZATION
    for (int i = 0; i < M; i++) {
      for (int j = i+1; j < M; j++) 
  cap[i][j] *= -1;
      b[i] *= -1;
    }
    c *= -1;
#endif

    for (int i = 0; i < M; i++) {
      if (b[i] >= 0) {
  cap[M][i] = b[i];
      } else {
  cap[i][M+1] = -b[i];
  c += b[i];
      }
    }

    int score = GetMaxFlow(M, M+1);
    fill(reached.begin(), reached.end(), 0);
    Augment(M, M+1, INF);
    x = VI(M);
    for (int i = 0; i < M; i++) x[i] = reached[i] ? 0 : 1;
    score += c;
#ifdef MAXIMIZATION
    score *= -1;
#endif

    return score;
  }

};

int main() {

  // solver for "Cat vs. Dog" from NWERC 2008
  
  int numcases;
  cin >> numcases;
  for (int caseno = 0; caseno < numcases; caseno++) {
    int c, d, v;
    cin >> c >> d >> v;

    VVVVI phi(c+d, VVVI(c+d, VVI(2, VI(2))));
    VVI psi(c+d, VI(2));
    for (int i = 0; i < v; i++) {
      char p, q;
      int u, v;
      cin >> p >> u >> q >> v;
      u--; v--;
      if (p == 'C') {
  phi[u][c+v][0][0]++;
  phi[c+v][u][0][0]++;
      } else {
  phi[v][c+u][1][1]++;
  phi[c+u][v][1][1]++;
      }
    }
    
    GraphCutInference graph;
    VI x;
    cout << graph.DoInference(phi, psi, x) << endl;
  }

  return 0;
}

\end{lstlisting}
\section{Geometry}
\subsection{Convex hull (C++)}
\begin{lstlisting}[language=C++]
// Compute the 2D convex hull of a set of points using the monotone chain
// algorithm.  Eliminate redundant points from the hull if REMOVE_REDUNDANT is 
// #defined.
//
// Running time: O(n log n)
//
//   INPUT:   a vector of input points, unordered.
//   OUTPUT:  a vector of points in the convex hull, counterclockwise, starting
//            with bottommost/leftmost point

#include <cstdio>
#include <cassert>
#include <vector>
#include <algorithm>
#include <cmath>

using namespace std;

#define REMOVE_REDUNDANT

typedef double T;
const T EPS = 1e-7;
struct PT { 
  T x, y; 
  PT() {} 
  PT(T x, T y) : x(x), y(y) {}
  bool operator<(const PT &rhs) const { return make_pair(y,x) < make_pair(rhs.y,rhs.x); }
  bool operator==(const PT &rhs) const { return make_pair(y,x) == make_pair(rhs.y,rhs.x); }
};

T cross(PT p, PT q) { return p.x*q.y-p.y*q.x; }
T area2(PT a, PT b, PT c) { return cross(a,b) + cross(b,c) + cross(c,a); }

#ifdef REMOVE_REDUNDANT
bool between(const PT &a, const PT &b, const PT &c) {
  return (fabs(area2(a,b,c)) < EPS && (a.x-b.x)*(c.x-b.x) <= 0 && (a.y-b.y)*(c.y-b.y) <= 0);
}
#endif

void ConvexHull(vector<PT> &pts) {
  sort(pts.begin(), pts.end());
  pts.erase(unique(pts.begin(), pts.end()), pts.end());
  vector<PT> up, dn;
  for (int i = 0; i < pts.size(); i++) {
    while (up.size() > 1 && area2(up[up.size()-2], up.back(), pts[i]) >= 0) up.pop_back();
    while (dn.size() > 1 && area2(dn[dn.size()-2], dn.back(), pts[i]) <= 0) dn.pop_back();
    up.push_back(pts[i]);
    dn.push_back(pts[i]);
  }
  pts = dn;
  for (int i = (int) up.size() - 2; i >= 1; i--) pts.push_back(up[i]);
  
#ifdef REMOVE_REDUNDANT
  if (pts.size() <= 2) return;
  dn.clear();
  dn.push_back(pts[0]);
  dn.push_back(pts[1]);
  for (int i = 2; i < pts.size(); i++) {
    if (between(dn[dn.size()-2], dn[dn.size()-1], pts[i])) dn.pop_back();
    dn.push_back(pts[i]);
  }
  if (dn.size() >= 3 && between(dn.back(), dn[0], dn[1])) {
    dn[0] = dn.back();
    dn.pop_back();
  }
  pts = dn;
#endif
}

\end{lstlisting}
\subsection{Miscellaneous geometry (C++)}
\begin{lstlisting}[language=C++]
// C++ routines for computational geometry.

#include <iostream>
#include <vector>
#include <cmath>
#include <cassert>

using namespace std;

double INF = 1e100;
double EPS = 1e-12;

struct PT { 
  double x, y; 
  PT() {}
  PT(double x, double y) : x(x), y(y) {}
  PT(const PT &p) : x(p.x), y(p.y)    {}
  PT operator + (const PT &p)  const { return PT(x+p.x, y+p.y); }
  PT operator - (const PT &p)  const { return PT(x-p.x, y-p.y); }
  PT operator * (double c)     const { return PT(x*c,   y*c  ); }
  PT operator / (double c)     const { return PT(x/c,   y/c  ); }
};

double dot(PT p, PT q)     { return p.x*q.x+p.y*q.y; }
double dist2(PT p, PT q)   { return dot(p-q,p-q); }
double cross(PT p, PT q)   { return p.x*q.y-p.y*q.x; }
ostream &operator<<(ostream &os, const PT &p) {
  os << "(" << p.x << "," << p.y << ")"; 
}

// rotate a point CCW or CW around the origin
PT RotateCCW90(PT p)   { return PT(-p.y,p.x); }
PT RotateCW90(PT p)    { return PT(p.y,-p.x); }
PT RotateCCW(PT p, double t) { 
  return PT(p.x*cos(t)-p.y*sin(t), p.x*sin(t)+p.y*cos(t)); 
}

// project point c onto line through a and b
// assuming a != b
PT ProjectPointLine(PT a, PT b, PT c) {
  return a + (b-a)*dot(c-a, b-a)/dot(b-a, b-a);
}

// project point c onto line segment through a and b
PT ProjectPointSegment(PT a, PT b, PT c) {
  double r = dot(b-a,b-a);
  if (fabs(r) < EPS) return a;
  r = dot(c-a, b-a)/r;
  if (r < 0) return a;
  if (r > 1) return b;
  return a + (b-a)*r;
}

// compute distance from c to segment between a and b
double DistancePointSegment(PT a, PT b, PT c) {
  return sqrt(dist2(c, ProjectPointSegment(a, b, c)));
}

// compute distance between point (x,y,z) and plane ax+by+cz=d
double DistancePointPlane(double x, double y, double z,
                          double a, double b, double c, double d)
{
  return fabs(a*x+b*y+c*z-d)/sqrt(a*a+b*b+c*c);
}

// determine if lines from a to b and c to d are parallel or collinear
bool LinesParallel(PT a, PT b, PT c, PT d) { 
  return fabs(cross(b-a, c-d)) < EPS; 
}

bool LinesCollinear(PT a, PT b, PT c, PT d) { 
  return LinesParallel(a, b, c, d)
      && fabs(cross(a-b, a-c)) < EPS
      && fabs(cross(c-d, c-a)) < EPS; 
}

// determine if line segment from a to b intersects with 
// line segment from c to d
bool SegmentsIntersect(PT a, PT b, PT c, PT d) {
  if (LinesCollinear(a, b, c, d)) {
    if (dist2(a, c) < EPS || dist2(a, d) < EPS ||
      dist2(b, c) < EPS || dist2(b, d) < EPS) return true;
    if (dot(c-a, c-b) > 0 && dot(d-a, d-b) > 0 && dot(c-b, d-b) > 0)
      return false;
    return true;
  }
  if (cross(d-a, b-a) * cross(c-a, b-a) > 0) return false;
  if (cross(a-c, d-c) * cross(b-c, d-c) > 0) return false;
  return true;
}

// compute intersection of line passing through a and b
// with line passing through c and d, assuming that unique
// intersection exists; for segment intersection, check if
// segments intersect first
PT ComputeLineIntersection(PT a, PT b, PT c, PT d) {
  b=b-a; d=c-d; c=c-a;
  assert(dot(b, b) > EPS && dot(d, d) > EPS);
  return a + b*cross(c, d)/cross(b, d);
}

// compute center of circle given three points
PT ComputeCircleCenter(PT a, PT b, PT c) {
  b=(a+b)/2;
  c=(a+c)/2;
  return ComputeLineIntersection(b, b+RotateCW90(a-b), c, c+RotateCW90(a-c));
}

// determine if point is in a possibly non-convex polygon (by William
// Randolph Franklin); returns 1 for strictly interior points, 0 for
// strictly exterior points, and 0 or 1 for the remaining points.
// Note that it is possible to convert this into an *exact* test using
// integer arithmetic by taking care of the division appropriately
// (making sure to deal with signs properly) and then by writing exact
// tests for checking point on polygon boundary
bool PointInPolygon(const vector<PT> &p, PT q) {
  bool c = 0;
  for (int i = 0; i < p.size(); i++){
    int j = (i+1)%p.size();
    if ((p[i].y <= q.y && q.y < p[j].y || 
      p[j].y <= q.y && q.y < p[i].y) &&
      q.x < p[i].x + (p[j].x - p[i].x) * (q.y - p[i].y) / (p[j].y - p[i].y))
      c = !c;
  }
  return c;
}

// determine if point is on the boundary of a polygon
bool PointOnPolygon(const vector<PT> &p, PT q) {
  for (int i = 0; i < p.size(); i++)
    if (dist2(ProjectPointSegment(p[i], p[(i+1)%p.size()], q), q) < EPS)
      return true;
    return false;
}

// compute intersection of line through points a and b with
// circle centered at c with radius r > 0
vector<PT> CircleLineIntersection(PT a, PT b, PT c, double r) {
  vector<PT> ret;
  b = b-a;
  a = a-c;
  double A = dot(b, b);
  double B = dot(a, b);
  double C = dot(a, a) - r*r;
  double D = B*B - A*C;
  if (D < -EPS) return ret;
  ret.push_back(c+a+b*(-B+sqrt(D+EPS))/A);
  if (D > EPS)
    ret.push_back(c+a+b*(-B-sqrt(D))/A);
  return ret;
}

// compute intersection of circle centered at a with radius r
// with circle centered at b with radius R
vector<PT> CircleCircleIntersection(PT a, PT b, double r, double R) {
  vector<PT> ret;
  double d = sqrt(dist2(a, b));
  if (d > r+R || d+min(r, R) < max(r, R)) return ret;
  double x = (d*d-R*R+r*r)/(2*d);
  double y = sqrt(r*r-x*x);
  PT v = (b-a)/d;
  ret.push_back(a+v*x + RotateCCW90(v)*y);
  if (y > 0)
    ret.push_back(a+v*x - RotateCCW90(v)*y);
  return ret;
}

// This code computes the area or centroid of a (possibly nonconvex)
// polygon, assuming that the coordinates are listed in a clockwise or
// counterclockwise fashion.  Note that the centroid is often known as
// the "center of gravity" or "center of mass".
double ComputeSignedArea(const vector<PT> &p) {
  double area = 0;
  for(int i = 0; i < p.size(); i++) {
    int j = (i+1) % p.size();
    area += p[i].x*p[j].y - p[j].x*p[i].y;
  }
  return area / 2.0;
}

double ComputeArea(const vector<PT> &p) {
  return fabs(ComputeSignedArea(p));
}

PT ComputeCentroid(const vector<PT> &p) {
  PT c(0,0);
  double scale = 6.0 * ComputeSignedArea(p);
  for (int i = 0; i < p.size(); i++){
    int j = (i+1) % p.size();
    c = c + (p[i]+p[j])*(p[i].x*p[j].y - p[j].x*p[i].y);
  }
  return c / scale;
}

// tests whether or not a given polygon (in CW or CCW order) is simple
bool IsSimple(const vector<PT> &p) {
  for (int i = 0; i < p.size(); i++) {
    for (int k = i+1; k < p.size(); k++) {
      int j = (i+1) % p.size();
      int l = (k+1) % p.size();
      if (i == l || j == k) continue;
      if (SegmentsIntersect(p[i], p[j], p[k], p[l])) 
        return false;
    }
  }
  return true;
}

int main() {
  
  // expected: (-5,2)
  cerr << RotateCCW90(PT(2,5)) << endl;
  
  // expected: (5,-2)
  cerr << RotateCW90(PT(2,5)) << endl;
  
  // expected: (-5,2)
  cerr << RotateCCW(PT(2,5),M_PI/2) << endl;
  
  // expected: (5,2)
  cerr << ProjectPointLine(PT(-5,-2), PT(10,4), PT(3,7)) << endl;
  
  // expected: (5,2) (7.5,3) (2.5,1)
  cerr << ProjectPointSegment(PT(-5,-2), PT(10,4), PT(3,7)) << " "
       << ProjectPointSegment(PT(7.5,3), PT(10,4), PT(3,7)) << " "
       << ProjectPointSegment(PT(-5,-2), PT(2.5,1), PT(3,7)) << endl;
  
  // expected: 6.78903
  cerr << DistancePointPlane(4,-4,3,2,-2,5,-8) << endl;
  
  // expected: 1 0 1
  cerr << LinesParallel(PT(1,1), PT(3,5), PT(2,1), PT(4,5)) << " "
       << LinesParallel(PT(1,1), PT(3,5), PT(2,0), PT(4,5)) << " "
       << LinesParallel(PT(1,1), PT(3,5), PT(5,9), PT(7,13)) << endl;
  
  // expected: 0 0 1
  cerr << LinesCollinear(PT(1,1), PT(3,5), PT(2,1), PT(4,5)) << " "
       << LinesCollinear(PT(1,1), PT(3,5), PT(2,0), PT(4,5)) << " "
       << LinesCollinear(PT(1,1), PT(3,5), PT(5,9), PT(7,13)) << endl;
  
  // expected: 1 1 1 0
  cerr << SegmentsIntersect(PT(0,0), PT(2,4), PT(3,1), PT(-1,3)) << " "
       << SegmentsIntersect(PT(0,0), PT(2,4), PT(4,3), PT(0,5)) << " "
       << SegmentsIntersect(PT(0,0), PT(2,4), PT(2,-1), PT(-2,1)) << " "
       << SegmentsIntersect(PT(0,0), PT(2,4), PT(5,5), PT(1,7)) << endl;
  
  // expected: (1,2)
  cerr << ComputeLineIntersection(PT(0,0), PT(2,4), PT(3,1), PT(-1,3)) << endl;
  
  // expected: (1,1)
  cerr << ComputeCircleCenter(PT(-3,4), PT(6,1), PT(4,5)) << endl;
  
  vector<PT> v; 
  v.push_back(PT(0,0));
  v.push_back(PT(5,0));
  v.push_back(PT(5,5));
  v.push_back(PT(0,5));
  
  // expected: 1 1 1 0 0
  cerr << PointInPolygon(v, PT(2,2)) << " "
       << PointInPolygon(v, PT(2,0)) << " "
       << PointInPolygon(v, PT(0,2)) << " "
       << PointInPolygon(v, PT(5,2)) << " "
       << PointInPolygon(v, PT(2,5)) << endl;
  
  // expected: 0 1 1 1 1
  cerr << PointOnPolygon(v, PT(2,2)) << " "
       << PointOnPolygon(v, PT(2,0)) << " "
       << PointOnPolygon(v, PT(0,2)) << " "
       << PointOnPolygon(v, PT(5,2)) << " "
       << PointOnPolygon(v, PT(2,5)) << endl;
  
  // expected: (1,6)
  //           (5,4) (4,5)
  //           blank line
  //           (4,5) (5,4)
  //           blank line
  //           (4,5) (5,4)
  vector<PT> u = CircleLineIntersection(PT(0,6), PT(2,6), PT(1,1), 5);
  for (int i = 0; i < u.size(); i++) cerr << u[i] << " "; cerr << endl;
  u = CircleLineIntersection(PT(0,9), PT(9,0), PT(1,1), 5);
  for (int i = 0; i < u.size(); i++) cerr << u[i] << " "; cerr << endl;
  u = CircleCircleIntersection(PT(1,1), PT(10,10), 5, 5);
  for (int i = 0; i < u.size(); i++) cerr << u[i] << " "; cerr << endl;
  u = CircleCircleIntersection(PT(1,1), PT(8,8), 5, 5);
  for (int i = 0; i < u.size(); i++) cerr << u[i] << " "; cerr << endl;
  u = CircleCircleIntersection(PT(1,1), PT(4.5,4.5), 10, sqrt(2.0)/2.0);
  for (int i = 0; i < u.size(); i++) cerr << u[i] << " "; cerr << endl;
  u = CircleCircleIntersection(PT(1,1), PT(4.5,4.5), 5, sqrt(2.0)/2.0);
  for (int i = 0; i < u.size(); i++) cerr << u[i] << " "; cerr << endl;
  
  // area should be 5.0
  // centroid should be (1.1666666, 1.166666)
  PT pa[] = { PT(0,0), PT(5,0), PT(1,1), PT(0,5) };
  vector<PT> p(pa, pa+4);
  PT c = ComputeCentroid(p);
  cerr << "Area: " << ComputeArea(p) << endl;
  cerr << "Centroid: " << c << endl;
  
  return 0;
}
\end{lstlisting}
\section{Numerical algorithms}
\subsection{Number theoretic algorithms (modular, Chinese remainder, linear Diophantine) (C++)}
\begin{lstlisting}[language=C++]
// This is a collection of useful code for solving problems that
// involve modular linear equations.  Note that all of the
// algorithms described here work on nonnegative integers.

#include <iostream>
#include <vector>
#include <algorithm>

using namespace std;

typedef vector<int> VI;
typedef pair<int,int> PII;

// return a % b (positive value)
int mod(int a, int b) {
  return ((a%b)+b)%b;
}

// computes gcd(a,b)
int gcd(int a, int b) {
  int tmp;
  while(b){a%=b; tmp=a; a=b; b=tmp;}
  return a;
}

// computes lcm(a,b)
int lcm(int a, int b) {
  return a/gcd(a,b)*b;
}

// returns d = gcd(a,b); finds x,y such that d = ax + by
int extended_euclid(int a, int b, int &x, int &y) {  
  int xx = y = 0;
  int yy = x = 1;
  while (b) {
    int q = a/b;
    int t = b; b = a%b; a = t;
    t = xx; xx = x-q*xx; x = t;
    t = yy; yy = y-q*yy; y = t;
  }
  return a;
}

// finds all solutions to ax = b (mod n)
VI modular_linear_equation_solver(int a, int b, int n) {
  int x, y;
  VI solutions;
  int d = extended_euclid(a, n, x, y);
  if (!(b%d)) {
    x = mod (x*(b/d), n);
    for (int i = 0; i < d; i++)
      solutions.push_back(mod(x + i*(n/d), n));
  }
  return solutions;
}

// computes b such that ab = 1 (mod n), returns -1 on failure
int mod_inverse(int a, int n) {
  int x, y;
  int d = extended_euclid(a, n, x, y);
  if (d > 1) return -1;
  return mod(x,n);
}

// Chinese remainder theorem (special case): find z such that
// z % x = a, z % y = b.  Here, z is unique modulo M = lcm(x,y).
// Return (z,M).  On failure, M = -1.
PII chinese_remainder_theorem(int x, int a, int y, int b) {
  int s, t;
  int d = extended_euclid(x, y, s, t);
  if (a%d != b%d) return make_pair(0, -1);
  return make_pair(mod(s*b*x+t*a*y,x*y)/d, x*y/d);
}

// Chinese remainder theorem: find z such that
// z % x[i] = a[i] for all i.  Note that the solution is
// unique modulo M = lcm_i (x[i]).  Return (z,M).  On 
// failure, M = -1.  Note that we do not require the a[i]'s
// to be relatively prime.
PII chinese_remainder_theorem(const VI &x, const VI &a) {
  PII ret = make_pair(a[0], x[0]);
  for (int i = 1; i < x.size(); i++) {
    ret = chinese_remainder_theorem(ret.second, ret.first, x[i], a[i]);
    if (ret.second == -1) break;
  }
  return ret;
}

// computes x and y such that ax + by = c; on failure, x = y =-1
void linear_diophantine(int a, int b, int c, int &x, int &y) {
  int d = gcd(a,b);
  if (c%d) {
    x = y = -1;
  } else {
    x = c/d * mod_inverse(a/d, b/d);
    y = (c-a*x)/b;
  }
}

int main() {
  
  // expected: 2
  cout << gcd(14, 30) << endl;
  
  // expected: 2 -2 1
  int x, y;
  int d = extended_euclid(14, 30, x, y);
  cout << d << " " << x << " " << y << endl;
  
  // expected: 95 45
  VI sols = modular_linear_equation_solver(14, 30, 100);
  for (int i = 0; i < (int) sols.size(); i++) cout << sols[i] << " "; 
  cout << endl;
  
  // expected: 8
  cout << mod_inverse(8, 9) << endl;
  
  // expected: 23 56
  //           11 12
  int xs[] = {3, 5, 7, 4, 6};
  int as[] = {2, 3, 2, 3, 5};
  PII ret = chinese_remainder_theorem(VI (xs, xs+3), VI(as, as+3));
  cout << ret.first << " " << ret.second << endl;
  ret = chinese_remainder_theorem (VI(xs+3, xs+5), VI(as+3, as+5));
  cout << ret.first << " " << ret.second << endl;
  
  // expected: 5 -15
  linear_diophantine(7, 2, 5, x, y);
  cout << x << " " << y << endl;

}

\end{lstlisting}
\subsection{Systems of linear equations, matrix inverse, determinant (C++)}
\begin{lstlisting}[language=C++]
// Gauss-Jordan elimination with full pivoting.
//
// Uses:
//   (1) solving systems of linear equations (AX=B)
//   (2) inverting matrices (AX=I)
//   (3) computing determinants of square matrices
//
// Running time: O(n^3)
//
// INPUT:    a[][] = an nxn matrix
//           b[][] = an nxm matrix
//
// OUTPUT:   X      = an nxm matrix (stored in b[][])
//           A^{-1} = an nxn matrix (stored in a[][])
//           returns determinant of a[][]

#include <iostream>
#include <vector>
#include <cmath>

using namespace std;

const double EPS = 1e-10;

typedef vector<int> VI;
typedef double T;
typedef vector<T> VT;
typedef vector<VT> VVT;

T GaussJordan(VVT &a, VVT &b) {
  const int n = a.size();
  const int m = b[0].size();
  VI irow(n), icol(n), ipiv(n);
  T det = 1;

  for (int i = 0; i < n; i++) {
    int pj = -1, pk = -1;
    for (int j = 0; j < n; j++) if (!ipiv[j])
      for (int k = 0; k < n; k++) if (!ipiv[k])
  if (pj == -1 || fabs(a[j][k]) > fabs(a[pj][pk])) { pj = j; pk = k; }
    if (fabs(a[pj][pk]) < EPS) { cerr << "Matrix is singular." << endl; exit(0); }
    ipiv[pk]++;
    swap(a[pj], a[pk]);
    swap(b[pj], b[pk]);
    if (pj != pk) det *= -1;
    irow[i] = pj;
    icol[i] = pk;

    T c = 1.0 / a[pk][pk];
    det *= a[pk][pk];
    a[pk][pk] = 1.0;
    for (int p = 0; p < n; p++) a[pk][p] *= c;
    for (int p = 0; p < m; p++) b[pk][p] *= c;
    for (int p = 0; p < n; p++) if (p != pk) {
      c = a[p][pk];
      a[p][pk] = 0;
      for (int q = 0; q < n; q++) a[p][q] -= a[pk][q] * c;
      for (int q = 0; q < m; q++) b[p][q] -= b[pk][q] * c;      
    }
  }

  for (int p = n-1; p >= 0; p--) if (irow[p] != icol[p]) {
    for (int k = 0; k < n; k++) swap(a[k][irow[p]], a[k][icol[p]]);
  }

  return det;
}

int main() {
  const int n = 4;
  const int m = 2;
  double A[n][n] = { {1,2,3,4},{1,0,1,0},{5,3,2,4},{6,1,4,6} };
  double B[n][m] = { {1,2},{4,3},{5,6},{8,7} };
  VVT a(n), b(n);
  for (int i = 0; i < n; i++) {
    a[i] = VT(A[i], A[i] + n);
    b[i] = VT(B[i], B[i] + m);
  }
  
  double det = GaussJordan(a, b);
  
  // expected: 60  
  cout << "Determinant: " << det << endl;

  // expected: -0.233333 0.166667 0.133333 0.0666667
  //           0.166667 0.166667 0.333333 -0.333333
  //           0.233333 0.833333 -0.133333 -0.0666667
  //           0.05 -0.75 -0.1 0.2
  cout << "Inverse: " << endl;
  for (int i = 0; i < n; i++) {
    for (int j = 0; j < n; j++)
      cout << a[i][j] << ' ';
    cout << endl;
  }
  
  // expected: 1.63333 1.3
  //           -0.166667 0.5
  //           2.36667 1.7
  //           -1.85 -1.35
  cout << "Solution: " << endl;
  for (int i = 0; i < n; i++) {
    for (int j = 0; j < m; j++)
      cout << b[i][j] << ' ';
    cout << endl;
  }
}

\end{lstlisting}
\subsection{Reduced row echelon form, matrix rank (C++)}
\begin{lstlisting}[language=C++]
// Reduced row echelon form via Gauss-Jordan elimination 
// with partial pivoting.  This can be used for computing
// the rank of a matrix.
//
// Running time: O(n^3)
//
// INPUT:    a[][] = an nxm matrix
//
// OUTPUT:   rref[][] = an nxm matrix (stored in a[][])
//           returns rank of a[][]

#include <iostream>
#include <vector>
#include <cmath>

using namespace std;

const double EPSILON = 1e-10;

typedef double T;
typedef vector<T> VT;
typedef vector<VT> VVT;

int rref(VVT &a) {
  int n = a.size();
  int m = a[0].size();
  int r = 0;
  for (int c = 0; c < m && r < n; c++) {
    int j = r;
    for (int i = r + 1; i < n; i++)
      if (fabs(a[i][c]) > fabs(a[j][c])) j = i;
    if (fabs(a[j][c]) < EPSILON) continue;
    swap(a[j], a[r]);

    T s = 1.0 / a[r][c];
    for (int j = 0; j < m; j++) a[r][j] *= s;
    for (int i = 0; i < n; i++) if (i != r) {
      T t = a[i][c];
      for (int j = 0; j < m; j++) a[i][j] -= t * a[r][j];
    }
    r++;
  }
  return r;
}

int main() {
  const int n = 5, m = 4;
  double A[n][m] = {
    {16,  2,  3, 13},
    { 5, 11, 10,  8},
    { 9,  7,  6, 12},
    { 4, 14, 15,  1},
    {13, 21, 21, 13}};
  VVT a(n);
  for (int i = 0; i < n; i++)
    a[i] = VT(A[i], A[i] + m);

  int rank = rref(a);

  // expected: 3
  cout << "Rank: " << rank << endl;

  // expected: 1 0 0 1 
  //           0 1 0 3 
  //           0 0 1 -3 
  //           0 0 0 3.10862e-15
  //           0 0 0 2.22045e-15
  cout << "rref: " << endl;
  for (int i = 0; i < 5; i++) {
    for (int j = 0; j < 4; j++)
      cout << a[i][j] << ' ';
    cout << endl;
  }
}

\end{lstlisting}
\subsection{Fast Fourier transform (C++)}
\begin{lstlisting}[language=C++]
#include <cassert>
#include <cstdio>
#include <cmath>

struct cpx
{
  cpx(){}
  cpx(double aa):a(aa),b(0){}
  cpx(double aa, double bb):a(aa),b(bb){}
  double a;
  double b;
  double modsq(void) const
  {
    return a * a + b * b;
  }
  cpx bar(void) const
  {
    return cpx(a, -b);
  }
};

cpx operator +(cpx a, cpx b)
{
  return cpx(a.a + b.a, a.b + b.b);
}

cpx operator *(cpx a, cpx b)
{
  return cpx(a.a * b.a - a.b * b.b, a.a * b.b + a.b * b.a);
}

cpx operator /(cpx a, cpx b)
{
  cpx r = a * b.bar();
  return cpx(r.a / b.modsq(), r.b / b.modsq());
}

cpx EXP(double theta)
{
  return cpx(cos(theta),sin(theta));
}

const double two_pi = 4 * acos(0);

// in:     input array
// out:    output array
// step:   {SET TO 1} (used internally)
// size:   length of the input/output {MUST BE A POWER OF 2}
// dir:    either plus or minus one (direction of the FFT)
// RESULT: out[k] = \sum_{j=0}^{size - 1} in[j] * exp(dir * 2pi * i * j * k / size)
void FFT(cpx *in, cpx *out, int step, int size, int dir)
{
  if(size < 1) return;
  if(size == 1)
  {
    out[0] = in[0];
    return;
  }
  FFT(in, out, step * 2, size / 2, dir);
  FFT(in + step, out + size / 2, step * 2, size / 2, dir);
  for(int i = 0 ; i < size / 2 ; i++)
  {
    cpx even = out[i];
    cpx odd = out[i + size / 2];
    out[i] = even + EXP(dir * two_pi * i / size) * odd;
    out[i + size / 2] = even + EXP(dir * two_pi * (i + size / 2) / size) * odd;
  }
}

// Usage:
// f[0...N-1] and g[0..N-1] are numbers
// Want to compute the convolution h, defined by
// h[n] = sum of f[k]g[n-k] (k = 0, ..., N-1).
// Here, the index is cyclic; f[-1] = f[N-1], f[-2] = f[N-2], etc.
// Let F[0...N-1] be FFT(f), and similarly, define G and H.
// The convolution theorem says H[n] = F[n]G[n] (element-wise product).
// To compute h[] in O(N log N) time, do the following:
//   1. Compute F and G (pass dir = 1 as the argument).
//   2. Get H by element-wise multiplying F and G.
//   3. Get h by taking the inverse FFT (use dir = -1 as the argument)
//      and *dividing by N*. DO NOT FORGET THIS SCALING FACTOR.

int main(void)
{
  printf("If rows come in identical pairs, then everything works.\n");
  
  cpx a[8] = {0, 1, cpx(1,3), cpx(0,5), 1, 0, 2, 0};
  cpx b[8] = {1, cpx(0,-2), cpx(0,1), 3, -1, -3, 1, -2};
  cpx A[8];
  cpx B[8];
  FFT(a, A, 1, 8, 1);
  FFT(b, B, 1, 8, 1);
  
  for(int i = 0 ; i < 8 ; i++)
  {
    printf("%7.2lf%7.2lf", A[i].a, A[i].b);
  }
  printf("\n");
  for(int i = 0 ; i < 8 ; i++)
  {
    cpx Ai(0,0);
    for(int j = 0 ; j < 8 ; j++)
    {
      Ai = Ai + a[j] * EXP(j * i * two_pi / 8);
    }
    printf("%7.2lf%7.2lf", Ai.a, Ai.b);
  }
  printf("\n");
  
  cpx AB[8];
  for(int i = 0 ; i < 8 ; i++)
    AB[i] = A[i] * B[i];
  cpx aconvb[8];
  FFT(AB, aconvb, 1, 8, -1);
  for(int i = 0 ; i < 8 ; i++)
    aconvb[i] = aconvb[i] / 8;
  for(int i = 0 ; i < 8 ; i++)
  {
    printf("%7.2lf%7.2lf", aconvb[i].a, aconvb[i].b);
  }
  printf("\n");
  for(int i = 0 ; i < 8 ; i++)
  {
    cpx aconvbi(0,0);
    for(int j = 0 ; j < 8 ; j++)
    {
      aconvbi = aconvbi + a[j] * b[(8 + i - j) % 8];
    }
    printf("%7.2lf%7.2lf", aconvbi.a, aconvbi.b);
  }
  printf("\n");
  
  return 0;
}

\end{lstlisting}
\subsection{Simplex algorithm (C++)}
\begin{lstlisting}[language=C++]
// Two-phase simplex algorithm for solving linear programs of the form
//
//     maximize     c^T x
//     subject to   Ax <= b
//                  x >= 0
//
// INPUT: A -- an m x n matrix
//        b -- an m-dimensional vector
//        c -- an n-dimensional vector
//        x -- a vector where the optimal solution will be stored
//
// OUTPUT: value of the optimal solution (infinity if unbounded
//         above, nan if infeasible)
//
// To use this code, create an LPSolver object with A, b, and c as
// arguments.  Then, call Solve(x).

#include <iostream>
#include <iomanip>
#include <vector>
#include <cmath>
#include <limits>

using namespace std;

typedef long double DOUBLE;
typedef vector<DOUBLE> VD;
typedef vector<VD> VVD;
typedef vector<int> VI;

const DOUBLE EPS = 1e-9;

struct LPSolver {
  int m, n;
  VI B, N;
  VVD D;

  LPSolver(const VVD &A, const VD &b, const VD &c) :
    m(b.size()), n(c.size()), N(n + 1), B(m), D(m + 2, VD(n + 2)) {
    for (int i = 0; i < m; i++) for (int j = 0; j < n; j++) D[i][j] = A[i][j];
    for (int i = 0; i < m; i++) { B[i] = n + i; D[i][n] = -1; D[i][n + 1] = b[i]; }
    for (int j = 0; j < n; j++) { N[j] = j; D[m][j] = -c[j]; }
    N[n] = -1; D[m + 1][n] = 1;
  }

  void Pivot(int r, int s) {
    for (int i = 0; i < m + 2; i++) if (i != r)
      for (int j = 0; j < n + 2; j++) if (j != s)
        D[i][j] -= D[r][j] * D[i][s] / D[r][s];
    for (int j = 0; j < n + 2; j++) if (j != s) D[r][j] /= D[r][s];
    for (int i = 0; i < m + 2; i++) if (i != r) D[i][s] /= -D[r][s];
    D[r][s] = 1.0 / D[r][s];
    swap(B[r], N[s]);
  }

  bool Simplex(int phase) {
    int x = phase == 1 ? m + 1 : m;
    while (true) {
      int s = -1;
      for (int j = 0; j <= n; j++) {
        if (phase == 2 && N[j] == -1) continue;
        if (s == -1 || D[x][j] < D[x][s] || D[x][j] == D[x][s] && N[j] < N[s]) s = j;
      }
      if (D[x][s] > -EPS) return true;
      int r = -1;
      for (int i = 0; i < m; i++) {
        if (D[i][s] < EPS) continue;
        if (r == -1 || D[i][n + 1] / D[i][s] < D[r][n + 1] / D[r][s] ||
          (D[i][n + 1] / D[i][s]) == (D[r][n + 1] / D[r][s]) && B[i] < B[r]) r = i;
      }
      if (r == -1) return false;
      Pivot(r, s);
    }
  }

  DOUBLE Solve(VD &x) {
    int r = 0;
    for (int i = 1; i < m; i++) if (D[i][n + 1] < D[r][n + 1]) r = i;
    if (D[r][n + 1] < -EPS) {
      Pivot(r, n);
      if (!Simplex(1) || D[m + 1][n + 1] < -EPS) return -numeric_limits<DOUBLE>::infinity();
      for (int i = 0; i < m; i++) if (B[i] == -1) {
        int s = -1;
        for (int j = 0; j <= n; j++)
          if (s == -1 || D[i][j] < D[i][s] || D[i][j] == D[i][s] && N[j] < N[s]) s = j;
        Pivot(i, s);
      }
    }
    if (!Simplex(2)) return numeric_limits<DOUBLE>::infinity();
    x = VD(n);
    for (int i = 0; i < m; i++) if (B[i] < n) x[B[i]] = D[i][n + 1];
    return D[m][n + 1];
  }
};

int main() {

  const int m = 4;
  const int n = 3;
  DOUBLE _A[m][n] = {
    { 6, -1, 0 },
    { -1, -5, 0 },
    { 1, 5, 1 },
    { -1, -5, -1 }
  };
  DOUBLE _b[m] = { 10, -4, 5, -5 };
  DOUBLE _c[n] = { 1, -1, 0 };

  VVD A(m);
  VD b(_b, _b + m);
  VD c(_c, _c + n);
  for (int i = 0; i < m; i++) A[i] = VD(_A[i], _A[i] + n);

  LPSolver solver(A, b, c);
  VD x;
  DOUBLE value = solver.Solve(x);

  cerr << "VALUE: " << value << endl; // VALUE: 1.29032
  cerr << "SOLUTION:"; // SOLUTION: 1.74194 0.451613 1
  for (size_t i = 0; i < x.size(); i++) cerr << " " << x[i];
  cerr << endl;
  return 0;
}

\end{lstlisting}
\section{Graph algorithms}
\subsection{Fast Dijkstra's algorithm (C++)}
\begin{lstlisting}[language=C++]
// Implementation of Dijkstra's algorithm using adjacency lists
// and priority queue for efficiency.
//
// Running time: O(|E| log |V|)

#include <queue>
#include <stdio.h>

using namespace std;
const int INF = 2000000000;
typedef pair<int,int> PII;

int main(){
  
  int N, s, t;
  scanf ("%d%d%d", &N, &s, &t);
  vector<vector<PII> > edges(N);
  for (int i = 0; i < N; i++){
    int M;
    scanf ("%d", &M);
    for (int j = 0; j < M; j++){
      int vertex, dist;
      scanf ("%d%d", &vertex, &dist);
      edges[i].push_back (make_pair (dist, vertex)); // note order of arguments here
    }
  }
  
  // use priority queue in which top element has the "smallest" priority
  priority_queue<PII, vector<PII>, greater<PII> > Q;
  vector<int> dist(N, INF), dad(N, -1);
  Q.push (make_pair (0, s));
  dist[s] = 0;
  while (!Q.empty()){
    PII p = Q.top();
    if (p.second == t) break;
    Q.pop();
    
    int here = p.second;
    for (vector<PII>::iterator it=edges[here].begin(); it!=edges[here].end(); it++){
      if (dist[here] + it->first < dist[it->second]){
        dist[it->second] = dist[here] + it->first;
        dad[it->second] = here;
        Q.push (make_pair (dist[it->second], it->second));
      }
    }
  }
  
  printf ("%d\n", dist[t]);
  if (dist[t] < INF)
    for(int i=t;i!=-1;i=dad[i])
      printf ("%d%c", i, (i==s?'\n':' '));
    
  return 0;
}

\end{lstlisting}
\subsection{Strongly connected components (C)}
\begin{lstlisting}[language=C++]
#include<memory.h>
struct edge{int e, nxt;};
int V, E;
edge e[MAXE], er[MAXE];
int sp[MAXV], spr[MAXV];
int group_cnt, group_num[MAXV];
bool v[MAXV];
int stk[MAXV];
void fill_forward(int x)
{
  int i;
  v[x]=true;
  for(i=sp[x];i;i=e[i].nxt) if(!v[e[i].e]) fill_forward(e[i].e);
  stk[++stk[0]]=x;
}
void fill_backward(int x)
{
  int i;
  v[x]=false;
  group_num[x]=group_cnt;
  for(i=spr[x];i;i=er[i].nxt) if(v[er[i].e]) fill_backward(er[i].e);
}
void add_edge(int v1, int v2) //add edge v1->v2
{
  e [++E].e=v2; e [E].nxt=sp [v1]; sp [v1]=E;
  er[  E].e=v1; er[E].nxt=spr[v2]; spr[v2]=E;
}
void SCC()
{
  int i;
  stk[0]=0;
  memset(v, false, sizeof(v));
  for(i=1;i<=V;i++) if(!v[i]) fill_forward(i);
  group_cnt=0;
  for(i=stk[0];i>=1;i--) if(v[stk[i]]){group_cnt++; fill_backward(stk[i]);}
}
\end{lstlisting}
\subsection{Eulerian Path (C++)}
\begin{lstlisting}
struct Edge;
typedef list<Edge>::iterator iter;

struct Edge
{
  int next_vertex;
  iter reverse_edge;

  Edge(int next_vertex)
    :next_vertex(next_vertex)
    { }
};

const int max_vertices = ;
int num_vertices;
list<Edge> adj[max_vertices];   // adjacency list

vector<int> path;

void find_path(int v)
{
  while(adj[v].size() > 0)
  {
    int vn = adj[v].front().next_vertex;
    adj[vn].erase(adj[v].front().reverse_edge);
    adj[v].pop_front();
    find_path(vn);
  }
  path.push_back(v);
}

void add_edge(int a, int b)
{
  adj[a].push_front(Edge(b));
  iter ita = adj[a].begin();
  adj[b].push_front(Edge(a));
  iter itb = adj[b].begin();
  ita->reverse_edge = itb;
  itb->reverse_edge = ita;
}

\end{lstlisting}
\section{Data structures}
\subsection{Suffix arrays (C++)}
\begin{lstlisting}[language=C++]
// Suffix array construction in O(L log^2 L) time.  Routine for
// computing the length of the longest common prefix of any two
// suffixes in O(log L) time.
//
// INPUT:   string s
//
// OUTPUT:  array suffix[] such that suffix[i] = index (from 0 to L-1)
//          of substring s[i...L-1] in the list of sorted suffixes.
//          That is, if we take the inverse of the permutation suffix[],
//          we get the actual suffix array.

#include <vector>
#include <iostream>
#include <string>

using namespace std;

struct SuffixArray {
  const int L;
  string s;
  vector<vector<int> > P;
  vector<pair<pair<int,int>,int> > M;

  SuffixArray(const string &s) : L(s.length()), s(s), P(1, vector<int>(L, 0)), M(L) {
    for (int i = 0; i < L; i++) P[0][i] = int(s[i]);
    for (int skip = 1, level = 1; skip < L; skip *= 2, level++) {
      P.push_back(vector<int>(L, 0));
      for (int i = 0; i < L; i++) 
  M[i] = make_pair(make_pair(P[level-1][i], i + skip < L ? P[level-1][i + skip] : -1000), i);
      sort(M.begin(), M.end());
      for (int i = 0; i < L; i++) 
  P[level][M[i].second] = (i > 0 && M[i].first == M[i-1].first) ? P[level][M[i-1].second] : i;
    }    
  }

  vector<int> GetSuffixArray() { return P.back(); }

  // returns the length of the longest common prefix of s[i...L-1] and s[j...L-1]
  int LongestCommonPrefix(int i, int j) {
    int len = 0;
    if (i == j) return L - i;
    for (int k = P.size() - 1; k >= 0 && i < L && j < L; k--) {
      if (P[k][i] == P[k][j]) {
  i += 1 << k;
  j += 1 << k;
  len += 1 << k;
      }
    }
    return len;
  }
};

int main() {

  // bobocel is the 0'th suffix
  //  obocel is the 5'th suffix
  //   bocel is the 1'st suffix
  //    ocel is the 6'th suffix
  //     cel is the 2'nd suffix
  //      el is the 3'rd suffix
  //       l is the 4'th suffix
  SuffixArray suffix("bobocel");
  vector<int> v = suffix.GetSuffixArray();
  
  // Expected output: 0 5 1 6 2 3 4
  //                  2
  for (int i = 0; i < v.size(); i++) cout << v[i] << " ";
  cout << endl;
  cout << suffix.LongestCommonPrefix(0, 2) << endl;
}

\end{lstlisting}
\subsection{Binary Indexed Tree}
\begin{lstlisting}[language=C++]
#include <iostream>
using namespace std;

#define LOGSZ 17

int tree[(1<<LOGSZ)+1];
int N = (1<<LOGSZ);

// add v to value at x
void set(int x, int v) {
  while(x <= N) {
    tree[x] += v;
    x += (x & -x);
  }
}

// get cumulative sum up to and including x
int get(int x) {
  int res = 0;
  while(x) {
    res += tree[x];
    x -= (x & -x);
  }
  return res;
}

// get largest value with cumulative sum less than or equal to x;
// for smallest, pass x-1 and add 1 to result
int getind(int x) {
  int idx = 0, mask = N;
  while(mask && idx < N) {
    int t = idx + mask;
    if(x >= tree[t]) {
      idx = t;
      x -= tree[t];
    }
    mask >>= 1;
  }
  return idx;
}

\end{lstlisting}
\subsection{Union-Find Set (C/C++)}
\begin{lstlisting}[language=C++]
//union-find set: the vector/array contains the parent of each node
int find(vector <int>& C, int x){return (C[x]==x) ? x : C[x]=find(C, C[x]);} //C++
int find(int x){return (C[x]==x)?x:C[x]=find(C[x]);} //C

\end{lstlisting}
\subsection{KD-tree (C++)}
\begin{lstlisting}[language=C++]
// -----------------------------------------------------------------
// A straightforward, but probably sub-optimal KD-tree implmentation
// that's probably good enough for most things (current it's a
// 2D-tree)
//
//  - constructs from n points in O(n lg^2 n) time
//  - handles nearest-neighbor query in O(lg n) if points are well
//    distributed
//  - worst case for nearest-neighbor may be linear in pathological
//    case
//
// Sonny Chan, Stanford University, April 2009
// -----------------------------------------------------------------
#include <iostream>
#include <vector>
#include <limits>
#include <cstdlib>
using namespace std;
// number type for coordinates, and its maximum value
typedef long long ntype;
const ntype sentry = numeric_limits<ntype>::max();
// point structure for 2D-tree, can be extended to 3D
struct point {
    ntype x, y;
    point(ntype xx = 0, ntype yy = 0) : x(xx), y(yy) {}
};
bool operator==(const point &a, const point &b) {
    return a.x == b.x && a.y == b.y;
}
// sorts points on x-coordinate
bool on_x(const point &a, const point &b) {
    return a.x < b.x;
}
// sorts points on y-coordinate
bool on_y(const point &a, const point &b) {
    return a.y < b.y;
}
// squared distance between points
ntype pdist2(const point &a, const point &b) {
    ntype dx = a.x-b.x, dy = a.y-b.y;
    return dx*dx + dy*dy;
}
// bounding box for a set of points
struct bbox {
    ntype x0, x1, y0, y1;
    
    bbox() : x0(sentry), x1(-sentry), y0(sentry), y1(-sentry) {}
    
    // computes bounding box from a bunch of points
    void compute(const vector<point> &v) {
        for (int i = 0; i < v.size(); ++i) {
            x0 = min(x0, v[i].x);   x1 = max(x1, v[i].x);
            y0 = min(y0, v[i].y);   y1 = max(y1, v[i].y);
        }
    }
    
    // squared distance between a point and this bbox, 0 if inside
    ntype distance(const point &p) {
        if (p.x < x0) {
            if (p.y < y0)       return pdist2(point(x0, y0), p);
            else if (p.y > y1)  return pdist2(point(x0, y1), p);
            else                return pdist2(point(x0, p.y), p);
        }
        else if (p.x > x1) {
            if (p.y < y0)       return pdist2(point(x1, y0), p);
            else if (p.y > y1)  return pdist2(point(x1, y1), p);
            else                return pdist2(point(x1, p.y), p);
        }
        else {
            if (p.y < y0)       return pdist2(point(p.x, y0), p);
            else if (p.y > y1)  return pdist2(point(p.x, y1), p);
            else                return 0;
        }
    }
};
// stores a single node of the kd-tree, either internal or leaf
struct kdnode {
    bool leaf;      // true if this is a leaf node (has one point)
    point pt;       // the single point of this is a leaf
    bbox bound;     // bounding box for set of points in children
    
    kdnode *first, *second; // two children of this kd-node
    
    kdnode() : leaf(false), first(0), second(0) {}
    ~kdnode() { if (first) delete first; if (second) delete second; }
    
    // intersect a point with this node (returns squared distance)
    ntype intersect(const point &p) {
        return bound.distance(p);
    }
    
    // recursively builds a kd-tree from a given cloud of points
    void construct(vector<point> &vp)
    {
        // compute bounding box for points at this node
        bound.compute(vp);
        
        // if we're down to one point, then we're a leaf node
        if (vp.size() == 1) {
            leaf = true;
            pt = vp[0];
        }
        else {
            // split on x if the bbox is wider than high (not best heuristic...)
            if (bound.x1-bound.x0 >= bound.y1-bound.y0)
                sort(vp.begin(), vp.end(), on_x);
            // otherwise split on y-coordinate
            else
                sort(vp.begin(), vp.end(), on_y);
            
            // divide by taking half the array for each child
            // (not best performance if many duplicates in the middle)
            int half = vp.size()/2;
            vector<point> vl(vp.begin(), vp.begin()+half);
            vector<point> vr(vp.begin()+half, vp.end());
            first = new kdnode();   first->construct(vl);
            second = new kdnode();  second->construct(vr);            
        }
    }
};
// simple kd-tree class to hold the tree and handle queries
struct kdtree {
    kdnode *root;
    
    // constructs a kd-tree from a points (copied here, as it sorts them)
    kdtree(const vector<point> &vp) {
        vector<point> v(vp.begin(), vp.end());
        root = new kdnode();
        root->construct(v);
    }
    ~kdtree() { delete root; }
    
    // recursive search method returns squared distance to nearest point
    ntype search(kdnode *node, const point &p) {
        if (node->leaf) {
            // commented special case tells a point not to find itself
//            if (p == node->pt) return sentry;
//            else               
                return pdist2(p, node->pt);
        }
        
        ntype bfirst = node->first->intersect(p);
        ntype bsecond = node->second->intersect(p);
        
        // choose the side with the closest bounding box to search first
        // (note that the other side is also searched if needed)
        if (bfirst < bsecond) {
            ntype best = search(node->first, p);
            if (bsecond < best)
                best = min(best, search(node->second, p));
            return best;
        }
        else {
            ntype best = search(node->second, p);
            if (bfirst < best)
                best = min(best, search(node->first, p));
            return best;
        }
    }
    
    // squared distance to the nearest 
    ntype nearest(const point &p) {
        return search(root, p);
    }
};
int main() {
    // generate some random points for a kd-tree
    vector<point> vp;
    for (int i = 0; i < 100000; ++i) {
        vp.push_back(point(rand()%100000, rand()%100000));
    }
    kdtree tree(vp);
    // query some points
    for (int i = 0; i < 10; ++i) {
        point q(rand()%100000, rand()%100000);
        cout << "Closest squared distance to (" << q.x << ", " << q.y << ")"
             << " is " << tree.nearest(q) << endl;
    }
    return 0;
}
\end{lstlisting}
\subsection{Splay Tree (C++)}
\begin{lstlisting}[language=C++]
#include <cstdio>
#include <algorithm>
using namespace std;
const int N_MAX = 130010;
const int oo = 0x3f3f3f3f;
struct Node {
  Node *ch[2], *pre;
  int val, size;
  bool isTurned;
} nodePool[N_MAX], *null, *root;
Node *allocNode(int val) {
  static int freePos = 0;
  Node *x = &nodePool[freePos ++];
  x->val = val, x->isTurned = false;
  x->ch[0] = x->ch[1] = x->pre = null;
  x->size = 1;
  return x;
}
inline void update(Node *x) {
  x->size = x->ch[0]->size + x->ch[1]->size + 1;
}
inline void makeTurned(Node *x) {
  if(x == null)
    return;
  swap(x->ch[0], x->ch[1]);
  x->isTurned ^= 1;
}
inline void pushDown(Node *x) {
  if(x->isTurned)
  {
    makeTurned(x->ch[0]);
    makeTurned(x->ch[1]);
    x->isTurned ^= 1;
  }
}
inline void rotate(Node *x, int c) {
  Node *y = x->pre;
  x->pre = y->pre;
  if(y->pre != null)
    y->pre->ch[y == y->pre->ch[1]] = x;
  y->ch[!c] = x->ch[c];
  if(x->ch[c] != null)
    x->ch[c]->pre = y;
  x->ch[c] = y, y->pre = x;
  update(y);
  if(y == root)
    root = x;
}
void splay(Node *x, Node *p) {
  while(x->pre != p)
  {
    if(x->pre->pre == p)
      rotate(x, x == x->pre->ch[0]);
    else
    {
      Node *y = x->pre, *z = y->pre;
      if(y == z->ch[0])
      {
        if(x == y->ch[0])
          rotate(y, 1), rotate(x, 1);
        else
          rotate(x, 0), rotate(x, 1);
      }
      else
      {
        if(x == y->ch[1])
          rotate(y, 0), rotate(x, 0);
        else
          rotate(x, 1), rotate(x, 0);
      }
    }
  }
  update(x);
}
void select(int k, Node *fa) {
  Node *now = root;
  while(1) {
    pushDown(now);
    int tmp = now->ch[0]->size + 1;
    if(tmp == k)
      break;
    else if(tmp < k)
      now = now->ch[1], k -= tmp;
    else
      now = now->ch[0];
  }
  splay(now, fa);
}
Node *makeTree(Node *p, int l, int r) {
  if(l > r)
    return null;
  int mid = (l + r) / 2;
  Node *x = allocNode(mid);
  x->pre = p;
  x->ch[0] = makeTree(x, l, mid - 1);
  x->ch[1] = makeTree(x, mid + 1, r);
  update(x);
  return x;
}
int main() {
  int n, m;
  null = allocNode(0);
  null->size = 0;
  root = allocNode(0);
  root->ch[1] = allocNode(oo);
  root->ch[1]->pre = root;
  update(root);
  scanf("%d%d", &n, &m);
  root->ch[1]->ch[0] = makeTree(root->ch[1], 1, n);
  splay(root->ch[1]->ch[0], null);
  while(m --) {
    int a, b;
    scanf("%d%d", &a, &b);
    a ++, b ++;
    select(a - 1, null);
    select(b + 1, root);
    makeTurned(root->ch[1]->ch[0]);
  }
  for(int i = 1; i <= n; i ++) {
    select(i + 1, null);
    printf("%d ", root->val);
  }
}
\end{lstlisting}
\subsection{Lowest Common Ancestor (C++)}
\begin{lstlisting}[language=C++]
const int max_nodes, log_max_nodes;
int num_nodes, log_num_nodes, root;
vector<int> children[max_nodes];  // children[i] contains the children of node i
int A[max_nodes][log_max_nodes+1];  // A[i][j] is the 2^j-th ancestor of node i, or -1 if that ancestor does not exist
int L[max_nodes];     // L[i] is the distance between node i and the root
// floor of the binary logarithm of n
int lb(unsigned int n)
{
    if(n==0)
  return -1;
    int p = 0;
    if (n >= 1<<16) { n >>= 16; p += 16; }
    if (n >= 1<< 8) { n >>=  8; p +=  8; }
    if (n >= 1<< 4) { n >>=  4; p +=  4; }
    if (n >= 1<< 2) { n >>=  2; p +=  2; }
    if (n >= 1<< 1) {           p +=  1; }
    return p;
}
void DFS(int i, int l)
{
    L[i] = l;
    for(int j = 0; j < children[i].size(); j++)
  DFS(children[i][j], l+1);
}
int LCA(int p, int q) {
    // ensure node p is at least as deep as node q
    if(L[p] < L[q])
  swap(p, q);
    // "binary search" for the ancestor of node p situated on the same level as q
    for(int i = log_num_nodes; i >= 0; i--)
  if(L[p] - (1<<i) >= L[q])
      p = A[p][i];
    if(p == q)
  return p;
    // "binary search" for the LCA
    for(int i = log_num_nodes; i >= 0; i--)
  if(A[p][i] != -1 && A[p][i] != A[q][i])
  {
      p = A[p][i];
      q = A[q][i];
  }
    return A[p][0];
}
int main(int argc,char* argv[])
{
    // read num_nodes, the total number of nodes
    log_num_nodes=lb(num_nodes);
    for(int i = 0; i < num_nodes; i++)
    {
  int p;
  // read p, the parent of node i or -1 if node i is the root

  A[i][0] = p;
  if(p != -1)
      children[p].push_back(i);
  else
      root = i;
    }
    // precompute A using dynamic programming
    for(int j = 1; j <= log_num_nodes; j++)
  for(int i = 0; i < num_nodes; i++)
      if(A[i][j-1] != -1)
    A[i][j] = A[A[i][j-1]][j-1];
      else
    A[i][j] = -1;
    // precompute L
    DFS(root, 0);   
    return 0;
}
\end{lstlisting}
\section{Miscellaneous}
\subsection{Longest increasing subsequence (C++)}
\begin{lstlisting}[language=C++]
// Given a list of numbers of length n, this routine extracts a 
// longest increasing subsequence.
// Running time: O(n log n)
//   INPUT: a vector of integers
//   OUTPUT: a vector containing the longest increasing subsequence
#include <iostream>
#include <vector>
#include <algorithm>
using namespace std;
typedef vector<int> VI;
typedef pair<int,int> PII;
typedef vector<PII> VPII;
#define STRICTLY_INCREASNG
VI LongestIncreasingSubsequence(VI v) {
  VPII best;
  VI dad(v.size(), -1);
  for (int i = 0; i < v.size(); i++) {
#ifdef STRICTLY_INCREASNG
    PII item = make_pair(v[i], 0);
    VPII::iterator it = lower_bound(best.begin(), best.end(), item);
    item.second = i;
#else
    PII item = make_pair(v[i], i);
    VPII::iterator it = upper_bound(best.begin(), best.end(), item);
#endif
    if (it == best.end()) {
      dad[i] = (best.size() == 0 ? -1 : best.back().second);
      best.push_back(item);
    } else {
      dad[i] = dad[it->second];
      *it = item;
    }
  }
  VI ret;
  for (int i = best.back().second; i >= 0; i = dad[i])
    ret.push_back(v[i]);
  reverse(ret.begin(), ret.end());
  return ret;
}
\end{lstlisting}
\subsection{Dates (C++)}
\begin{lstlisting}[language=C++]
// Routines for performing computations on dates.  In these routines,
// months are expressed as integers from 1 to 12, days are expressed
// as integers from 1 to 31, and years are expressed as 4-digit
// integers.
#include <iostream>
#include <string>
using namespace std;
string dayOfWeek[] = {"Mon", "Tue", "Wed", "Thu", "Fri", "Sat", "Sun"};
// converts Gregorian date to integer (Julian day number)
int dateToInt (int m, int d, int y){  
  return 
    1461 * (y + 4800 + (m - 14) / 12) / 4 +
    367 * (m - 2 - (m - 14) / 12 * 12) / 12 - 
    3 * ((y + 4900 + (m - 14) / 12) / 100) / 4 + 
    d - 32075;
}
// converts integer (Julian day number) to Gregorian date: month/day/year
void intToDate (int jd, int &m, int &d, int &y){
  int x, n, i, j; 
  x = jd + 68569;
  n = 4 * x / 146097;
  x -= (146097 * n + 3) / 4;
  i = (4000 * (x + 1)) / 1461001;
  x -= 1461 * i / 4 - 31;
  j = 80 * x / 2447;
  d = x - 2447 * j / 80;
  x = j / 11;
  m = j + 2 - 12 * x;
  y = 100 * (n - 49) + i + x;
}
// converts integer (Julian day number) to day of week
string intToDay (int jd){
  return dayOfWeek[jd % 7];
}
int main (int argc, char **argv){
  int jd = dateToInt (3, 24, 2004);
  int m, d, y;
  intToDate (jd, m, d, y);
  string day = intToDay (jd);
  // expected output: 2453089 3/24/2004 Wed
  cout << jd << endl
    << m << "/" << d << "/" << y << endl
    << day << endl;
}
\end{lstlisting}
\subsection{Prime numbers (C++)}
\begin{lstlisting}[language=C++]
// O(sqrt(x)) Exhaustive Primality Test
#include <cmath>
#define EPS 1e-7
typedef long long LL;
bool IsPrimeSlow (LL x) {
  if(x<=1) return false;
  if(x<=3) return true;
  if (!(x%2) || !(x%3)) return false;
  LL s=(LL)(sqrt((double)(x))+EPS);
  for(LL i=5;i<=s;i+=6) {
    if (!(x%i) || !(x%(i+2))) return false;
  }
  return true;
}
\end{lstlisting}
\subsection{Latitude/longitude}
\begin{lstlisting}[language=C++]
/* Converts from rectangular coordinates to latitude/longitude and vice
versa. Uses degrees (not radians). */
#include <iostream>
#include <cmath>
using namespace std;
struct ll {
  double r, lat, lon;
};
struct rect {
  double x, y, z;
};
ll convert(rect& P) {
  ll Q;
  Q.r = sqrt(P.x*P.x+P.y*P.y+P.z*P.z);
  Q.lat = 180/M_PI*asin(P.z/Q.r);
  Q.lon = 180/M_PI*acos(P.x/sqrt(P.x*P.x+P.y*P.y));
  return Q;
}
rect convert(ll& Q) {
  rect P;
  P.x = Q.r*cos(Q.lon*M_PI/180)*cos(Q.lat*M_PI/180);
  P.y = Q.r*sin(Q.lon*M_PI/180)*cos(Q.lat*M_PI/180);
  P.z = Q.r*sin(Q.lat*M_PI/180);
  return P;
}
int main() {
  rect A;
  ll B;
  A.x = -1.0; A.y = 2.0; A.z = -3.0;
  B = convert(A);
  cout << B.r << " " << B.lat << " " << B.lon << endl;
  A = convert(B);
  cout << A.x << " " << A.y << " " << A.z << endl;
}

\end{lstlisting}
